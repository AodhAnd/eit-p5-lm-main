%implementing ocument formatting:
\input{preamble.tex}
\begin{document}
\renewcommand\chaptername{KAPITEL}
\renewcommand\contentsname{Indhold}
\renewcommand\figurename{Figur}
\renewcommand\tablename{Tabel}

\section*{Supervisor meeting\\ \small 27th of November 2015}

\subsection{Problem with SD card}
\begin{itemize}
\item[-] It is not so important, so fine to cut it because of time constraints.
\end{itemize}

\subsection{Magnetometer}
\begin{itemize}
\item[-] Calibration of the offset worked
\item[-] Fixed the incoherence problems due to external magnetic fields (wires, battery and other metallic objects moving around the sensor)
\end{itemize}

\subsection{Steering test}
\begin{itemize}
  \item[-] Good job with has been done for the steering test yet
  \item[-] A problem was found during the test; the steering is not consistent, maybe because of heating brakes
  \item[-] Maybe we are lucky, and the controller will fix the problem
  \item[-] Try making 10 test and see the difference between them
  \item[-] Just jump out to it : try to estimate, test it and adapt it accordingly
  \item[-] There should be a integrator in the steering model, which should result in a type 1 system
  \item[-] Integrator is depending on the velocity : use an operating point (assume constant speed)
  \item[-] The black box drawn yet looks correct 
  \item[-] The time constant should vary according to the velocity
  \item[-] When making this linear, the speed shall be kept constant. P controller
  \item[-] Don't use too much time on different types of controller
  \item[-] Make an inner loop to measure the angle and an outer loop the distance to a determined line
\end{itemize}

\subsection{Velocity part}
\begin{itemize}
\item[-] There are just few comments, but otherwis it is good
\item[-] Be consistent regarding the terms in equations: clarify which are constant wich are varying depending on time
\item[-] In Equation 5.20, the $2 \cdot \pi$ should not be there (same thing for other equations)
\item[-] $L_a$ can not just removed because it is small, but because the associated pole is much less important/far from the dominant pole/out of the range of passing frequencies for our system to show the difference between 1st and 2nd order
\item[-] In Figure 5.13 : 
  \begin{itemize}
  \item[-] Rounding errors have gave a velocity smaller than 0
  \item[-] More explanation about the delay in 5.13
  \end{itemize}
\item[-] Make a step down
\end{itemize}


\subsection{Signal processings}
\begin{itemize}
\item[-] If we really want to make a filter, make a really fast sampling on the hall sensor, so it could make it there
\end{itemize}


\subsection{Next Supervisor meeting}
Next Friday (4th of December) at 12.30

\end{document}

% TEGN
%-----------------------------
% Højrepil:		$\rightarrow$