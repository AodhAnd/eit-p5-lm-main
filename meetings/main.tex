%implementing ocument formatting:
\input{preamble.tex}
\begin{document}
\renewcommand\chaptername{KAPITEL}
\renewcommand\contentsname{Indhold}
\renewcommand\figurename{Figur}
\renewcommand\tablename{Tabel}

\section*{Supervisor meeting\\ \small 20th of November 2015}

\subsection{Comparison of Step-responses with controllers}
\begin{itemize}
\item[-] Done again with new gain and stiction offset
\item[-] P-controller with feed-forward is fine for straight runs of the vehicle
\item[-] It might be necessary to add an integrator when adding the steering
\end{itemize}

\subsection{Comparison of first and second order system simulations}
\begin{itemize}
\item[-] Done again with new gain and time constant $\rightarrow$ fine
\end{itemize}

\subsection{Inertia Formula}
\begin{itemize}
\item[-] Inertia formula fixed
\end{itemize}

\subsection{Steering Model And Test}
\begin{itemize}
\item[-] Magnetometer data: have to do a plot of the vehicle turning without motor, to see if it's the magnet wich is not good or if we juste have to do a filtering on the data because of the magnetic field from the motor.
\end{itemize}


\subsection{Hall Sensor Section}
\begin{itemize}
\item[-] Try to do a plot comparison between the  wanted and the real velocity, to have a statistic error.
\item[-] Algorithm flowchart should be in appendix
\item[-] Add in the report that the frequency sampling has to be constant.
\end{itemize}

\subsection{Requirements}
\begin{itemize}
\item[-] Velocity: the vehicle should be able to keep the velocity for different disturbance and inputs.
\item[-] Controller: should control the disturbances more than the input error, and in a certain time. It should ensure the stability of the system, even on steep terrain, for example.
\end{itemize}


\subsection{Route planning}
\begin{itemize}
\item[-] Shoud be simple, like a spiral, smooth
\end{itemize}


\subsection{Next Supervisor meeting}
Next Friday (27th of November) at 12.30

\end{document}

% TEGN
%-----------------------------
% Højrepil:		$\rightarrow$