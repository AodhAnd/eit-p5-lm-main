%implementing ocument formatting:
\input{preamble.tex}
\begin{document}
\renewcommand\chaptername{KAPITEL}
\renewcommand\contentsname{Indhold}
\renewcommand\figurename{Figur}
\renewcommand\tablename{Tabel}

\section*{Supervisor meeting\\ \small 13th of November 2015}
\subsection{Vehicle running free (without controller)}
\begin{itemize}
\item[-] There is a lot of Coulomb friction in this system (stiction at the beginning)
\item[-] Adapt the voltage at the beginning to get rid of this dry friction (put an offset) \\
$\rightarrow$ Then the first step might have the same gain and time constant as the second step, in which case the system would look closer to a first order system
\item[-] The comparison of step-responses between real and approximated systems seems really nice : they match very well
\end{itemize}

\subsection{Controllers}
\begin{itemize}
\item[-] both need to be re-done with the right gain which has to be re-measured with the right PWM frequency
\end{itemize}

\subsection{Steering model and tests}
\begin{itemize}
\item[-] The time must appear in the test
\item[-] We need to consider `small signals' : try a step-response by driving at a constant speed and then steer (use angle output)
\item[-] See how fast it integrates the angular velocity
\item[-] Do only the new test (just described)
\item[-] Maybe the modeling of the differential is not that complicated (2 equations) :
\begin{itemize}
  \item $\omega_m = (\omega_1 + \omega_2) \cdot K$
  \item $\tau_m \cdot \omega_m = (\tau_1 \cdot \omega_1 + \tau_2 \cdot \omega_2) + K^2$, where $K$ is gear ratio
\end{itemize}
\item[-] See how the measured data fits the modeling of the steering

\end{itemize}

\subsection{Inertia formula}
\begin{itemize}
\item[-] We need to look at it a bit more before Tom wil help us
\end{itemize}

\subsection{Next Supervisor meeting}
Next Friday (20th of November) at 12.30

\end{document}

% TEGN
%-----------------------------
% Højrepil:		$\rightarrow$