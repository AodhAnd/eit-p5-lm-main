$>$$>$$>$ K\+R\+N\+L -\/ a small preemptive kernel for small systems $<$$<$$<$

I have found it interesting to develop an open source realtime kernel

for the Arduino platform -\/ but is also portable to other platforms


\begin{DoxyItemize}
\item S\+E\+E S\+O\+M\+E N\+O\+T\+E\+S B\+E\+L\+O\+W A\+B\+O\+U\+T T\+I\+M\+E\+R\+S A\+N\+D P\+I\+N\+S
\item Now doxygen docu at html directory \+:-\/)
\item See \hyperlink{krnl_8h}{krnl.\+h} for further comments
\item -\/ timers
\item -\/ 8/16 M\+Hz setting
\item -\/ etc
\end{DoxyItemize}

\subsection*{Some highlights }


\begin{DoxyItemize}
\item open source (beer license)
\item well suited for teaching
\begin{DoxyItemize}
\item easy to read and understand source
\item priority to straight code instead insane optimization(which will make it nearly unreadable)
\end{DoxyItemize}
\item well suited for serious duties -\/ but with no warranty what so ever -\/ only use it at own risk !!!
\item easy to use
\begin{DoxyItemize}
\item just import library krnl and you are ready
\end{DoxyItemize}
\item automatic recognition of Arduino architeture
\begin{DoxyItemize}
\item supports all atmega variants I have had available (168,328,1280,2560 -\/ uno, duemillanove, mega 1280 and 2560) Some characteristics\+:
\end{DoxyItemize}
\item preemptive scheduling
\begin{DoxyItemize}
\item Basic heart beat at 1 k\+Hz. S\+N\+O\+T can have heeartbeat in quants of milli seconds
\item static priority scheme
\end{DoxyItemize}
\item support task, semaphores, message queues
\begin{DoxyItemize}
\item All elements shall be allocated prior to start of K\+R\+N\+L
\end{DoxyItemize}
\item support user I\+S\+Rs and external interrupts
\item timers
\begin{DoxyItemize}
\item krnl can be configures to use tmr 1,2 and for mega also 3,4,5 for running krnl tick
\item For timer 0 you should take care of millis and it will require some modifications in arduino lib
\item see \hyperlink{krnl_8h}{krnl.\+h} for implications (like
\end{DoxyItemize}
\item Accuracy
\begin{DoxyItemize}
\item 8 bit timers (0,2) 1 millisecond is 15.\+625 countdown on timer
\begin{DoxyItemize}
\item example 10 msec 156 instead of 156.\+25 so an error of 0.\+25/156.25 $\sim$= 0.\+2\%
\end{DoxyItemize}
\item 16 bit timers count down is 1 millisecond for 62.\+5 count
\item -\/ example 10 msec $\sim$ 625 countdown == precise \+:-\/)
\end{DoxyItemize}
\end{DoxyItemize}

See in \hyperlink{krnl_8h}{krnl.\+h} for information like ...

... from \href{http://blog.oscarliang.net/arduino-timer-and-interrupt-tutorial/}{\tt http\+://blog.\+oscarliang.\+net/arduino-\/timer-\/and-\/interrupt-\/tutorial/} Timer0\+:
\begin{DoxyItemize}
\item Timer0 and 2 is a 8bit timer.
\item In the Arduino world Timer0 is been used for the timer functions, like delay(), millis() and micros().
\item If you change Timer0 registers, this may influence the Arduino timer function.
\item So you should know what you are doing.
\end{DoxyItemize}

Timer1\+:
\begin{DoxyItemize}
\item Timer1 is a 16bit timer.
\item In the Arduino world the Servo library uses Timer1 on Arduino Uno (Timer5 on Arduino Mega).
\end{DoxyItemize}

Timer2\+:
\begin{DoxyItemize}
\item Timer2 is a 8bit timer like Timer0.
\item In the Arduino work the tone() function uses Timer2.
\end{DoxyItemize}

Timer3, Timer4, Timer5\+: Timer 3,4,5 are only available on Arduino Mega boards.
\begin{DoxyItemize}
\item These timers are all 16bit timers.
\end{DoxyItemize}

Install from github\+:

1) cd whatever/sketchbook/libraries -\/ see Preferences for path to sketchbook 2) git clone \href{https://github.com/jdn-aau/krnl.git}{\tt https\+://github.\+com/jdn-\/aau/krnl.\+git}

N\+B N\+B N\+B -\/ T\+I\+M\+E\+R H\+E\+A\+R\+T\+B\+E\+A\+T From vrs 1236 you can change which timer to use in \hyperlink{krnl_8c}{krnl.\+c} Just look in top of file for K\+R\+N\+L\+T\+M\+R
\begin{DoxyItemize}
\item tested with uno and mega 256
\end{DoxyItemize}

In \hyperlink{krnl_8c}{krnl.\+c} you can configure K\+R\+N\+L to use timer (0),1,2,3,4 or 5. (3,4,5 only for 1280/2560 mega variants)

You can select heartbeat between 1 and 200 milliseconds in 1 msec steps.


\begin{DoxyItemize}
\item Timer0 -\/ An 8 bit timer used by Arduino functions delay(), millis() and micros(). B\+E\+W\+A\+R\+E
\item Timer1 -\/ A 16 bit timer used by the Servo() library
\item Timer2 -\/ An 8 bit timer used by the Tone() library
\item Timer3,4,5 16 bits
\end{DoxyItemize}

... from \href{http://arduino-info.wikispaces.com/Timers-Arduino}{\tt http\+://arduino-\/info.\+wikispaces.\+com/\+Timers-\/\+Arduino}


\begin{DoxyItemize}
\item Servo Library uses Timer1.
\item -\/ You can’t use P\+W\+M on Pin 9, 10 when you use the Servo Library on an Arduino.
\item -\/ For Arduino Mega it is a bit more difficult. The timer needed depends on the number of servos.
\item -\/ Each timer can handle 12 servos.
\item -\/ For the first 12 servos timer 5 will be used (losing P\+W\+M on Pin 44,45,46).
\item -\/ For 24 Servos timer 5 and 1 will be used (losing P\+W\+M on Pin 11,12,44,45,46)..
\item -\/ For 36 servos timer 5, 1 and 3 will be used (losing P\+W\+M on Pin 2,3,5,11,12,44,45,46)..
\item -\/ For 48 servos all 16bit timers 5,1,3 and 4 will be used (losing all P\+W\+M pins).
\item Pin 11 has shared functionality P\+W\+M and M\+O\+S\+I.
\item -\/ M\+O\+S\+I is needed for the S\+P\+I interface, You can’t use P\+W\+M on Pin 11 and the S\+P\+I interface at the same time on Arduino.
\item -\/ On the Arduino Mega the S\+P\+I pins are on different pins.
\item tone() function uses at least timer2.
\item -\/ You can’t use P\+W\+M on Pin 3,11 when you use the tone() function an Arduino and Pin 9,10 on Arduino Mega.
\end{DoxyItemize}

(c)
\begin{DoxyItemize}
\item \char`\"{}\+T\+H\+E B\+E\+E\+R-\/\+W\+A\+R\+E L\+I\+C\+E\+N\+S\+E\char`\"{} (frit efter P\+H\+K) $\ast$
\begin{DoxyItemize}
\item \href{mailto:jdn@es.aau.dk}{\tt jdn@es.\+aau.\+dk} wrote this file. As long as you $\ast$
\item retain this notice you can do whatever you want $\ast$
\item with this stuff. If we meet some day, and you think$\ast$
\item this stuff is worth it ... $\ast$
\item you can buy me a beer in return \+:-\/) $\ast$
\item or if you are real happy then ... $\ast$
\item single malt will be well received \+:-\/) $\ast$
\item $\ast$
\item Use it at your own risk -\/ no warranty
\end{DoxyItemize}
\end{DoxyItemize}

Happy hacking

See also \href{http://es.aau.dk/staff/jdn/edu/doc/arduino/krnl}{\tt http\+://es.\+aau.\+dk/staff/jdn/edu/doc/arduino/krnl}

/\+Jens 